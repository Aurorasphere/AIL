\documentclass[12pt,oneside,a4paper]{oblivoir}
\usepackage[a4paper, margin=2cm]{geometry}
\usepackage{kotex}
\usepackage{hyperref}
\usepackage{enumitem}

\title{Syllabus: Introduction to Linguistics}
\date{Spring 2025 \\ 20:00~22:00 (KST) \\ Class Location: KSK Discord Server}
\author{}

\begin{document}
\maketitle

\section*{Class Information}
\begin{itemize}[leftmargin=2em]
    \item \textbf{Course Title:} Introduction to Linguistics
    \item \textbf{Schedule:} Twice a week (Every Monday and Thursday)
    \item \textbf{Location:} KSK Discord Server (aurorasphere)
    \item \textbf{Start Date:} Next week (May 19th, 2025)
\end{itemize}

\section*{Instructor}
\begin{itemize}[leftmargin=2em]
    \item \textbf{Name:} Dongjun "Aurorasphere" Kim 
    \item \textbf{Platform:} Discord (DM for contact), \textbf{Email} aurorasphere@o.cnu.ac.kr
    \item \textbf{Office Hours:} Flexible (by appointment via Discord and Email)
\end{itemize}

\section*{Course Description}
This course is a basic introduction to linguistics, designed specifically for programmers and coders. 
It covers the fundamental aspects of human language, including phonetics, phonology, morphology, syntax, semantics, and pragmatics. And students are going to look at some of the topics of applied linguistics.
Throughout the course, we will explore how human languages are structured and how these structures can be modeled using tools and concepts familiar to computer scientists. Special attention will be given to formal systems such as phrase structure grammars, feature structures, and the Chomsky hierarchy, all of which connect naturally to fields like compilers, parsers, and natural language processing (NLP).
By comparing natural language with programming languages, students will gain a new perspective on language as a rule-governed system, and develop analytical skills that can be applied both in linguistics and in computational contexts.

\section*{Course Objectives}

By the end of this course, students will be able to:

\begin{itemize}
    \item Understand the basic components of human language
    \item Identify the differences and similarities between natural and programming languages
    \item Analyze sentence structures using phrase structure rules and formal grammar
    \item Apply linguistic concepts to computational fields such as compilers and NLP
    \item Develop critical thinking and modeling skills based on language data
\end{itemize}

\section*{Prerequisites}
\begin{itemize}
    \item None! This is Aurora's first linguistics lecture, so no prerequisites are required.
\end{itemize}

\section*{Tentative Schedule}

\begin{itemize}
    \item \textbf{Week 1:} Introduction – What is Language? Why Linguistics for Programmers?
    \item \textbf{Week 2:} A Brief History of Linguistics – Key Concepts and Theoretical Foundations
    \item \textbf{Week 3:} Phonetics and Phonology – The Sounds of Human Language
    \item \textbf{Week 4:} Morphology: Word Formation and Morphemes
    \item \textbf{Week 5:} Syntax – Phrase Structure and Generative grammar
    \item \textbf{Week 6:} Semantics and Pragmatics – Meaning, Reference, and Use
    \item \textbf{Week 7:} Formal Language Theory – Chomsky Hierarchy, CFGs, PDAs
    \item \textbf{Week 8:} Applied Linguistics
    \item \textbf{Week 9:} NLP and Linguistics – Applications and Closing Discussion
\end{itemize}

\section*{Assessment}
\begin{itemize}
    \item \textbf{Weekly Quizzes:} A short quiz will be given every Thursday. These quizzes are completely optional, and will cover the material from the week's lectures and are intended to reinforce understanding rather than evaluate performance strictly.
    \item \textbf{Participation:} Active participation during discussions and Q\&A sessions is encouraged.
\end{itemize}

\section*{Recommended Materials}

\begin{itemize}
\item \textbf{In English:} Ohio State University, \textit{Language Files}
\item \textbf{In Korean:} 김진우, \textit{언어: 이론과 그 응용}; 강범모, \textit{언어: 풀어쓴 언어학 개론}

\end{itemize}

\end{document}

